%-*- Mode:LaTeX; -*-      
\documentclass[11pt]{article}
\usepackage{vmargin}		% Force narrower margins
\setpapersize{USletter}
\setmarginsrb{1.0in}{1.0in}{1.0in}{0.6in}{0pt}{0pt}{0pt}{0.4in}
\setlength{\parskip}{.1in}  % removed space between paragraphs
\setlength{\parindent}{0in}

\usepackage{epsfig}
\usepackage{graphicx}
\newcommand{\ra}{$\rightarrow$~}
\newcommand{\dt}{$\circ$~}

\begin{document}

\large
\begin{center}
{\bf CS-5340/6340, Written Assignment \#1} \\
{\bf DUE: Tuesday, September 4, 2018 by 11:59pm}
\end{center}
\normalsize

\begin{enumerate}  

%% ===============================================================
% QUESTION #1 : Parts of Speech    
%% =============================================================

\item (33 pts) For each sentence below, label each word with its
  correct part-of-speech (POS) tag based upon the word's use in the sentence.
  Do not assign POS tags to punctuation marks. 

  Choose from the following list of part-of-speech tags: {\bf
    adjective ({\sc adj}), adverb ({\sc adv}), article ({\sc art}),
    conjunction ({\sc conj}), gerund ({\sc ger}), infinitive ``to''
    ({\sc inf}), modal verb ({\sc mod}), noun ({\sc noun}), particle
    ({\sc part}), preposition ({\sc prep}), personal 
    pronoun ({\sc perpro}), relative pronoun ({\sc relpro}), verb
    ({\sc verb})} [not modal]. 

For infinitive verb phrase constructions, label ``to'' as {\sc inf}
and the verb itself as {\sc verb}. 

NOTE: An easy way to show your part-of-speech tags is to append a
slash and POS tag after each word. For example: ``Natural/{\sc adj}
language/{\sc noun} is/{\sc verb} fun/{\sc adj}.'' \\


\begin{enumerate}

\item John might go to school to learn welding.
\vspace*{.2in}


\item Fires broke out near Moab without warning
\vspace*{.2in}


\item Kate, who is brilliant, builds robots. 
\vspace*{.2in}



\item Susan loves her house but may not stay there.
\vspace*{.2in}



\item Sleeping bags can keep people from freezing on camping trips.
\vspace*{.2in}


\item To quit smoking, Joe never buys cigarettes, which he craves.
\vspace*{.2in}


\item  Tom makes up stories about imaginary monsters.
\vspace*{.2in}



\item She delicately brought up a sensitive topic to Bill.
\vspace*{.2in}


\end{enumerate}


\newpage
%% ===============================================================
% QUESTION #2: Active/Passive Voice
%% ================================================================

\item (20 pts) For each sentence below, indicate whether the main verb
  is in an {\it active voice} or {\it passive voice} construction. 

\begin{enumerate}

\item The bird quickly built an amazing nest for its young.
\vspace*{.5in}

\item George has been having serious problems with his back.
\vspace*{.5in}

\item Mary does not want help with her car.
\vspace*{.5in}

\item The window of the Toyota Prius was broken during a hail storm.
\vspace*{.5in}

\item Too much money has been spent on unnecessary trips to Europe.
\vspace*{.5in}

\item He could barely understand the writing on the chalkboard.
\vspace*{.5in}

\item Both girls from Utah were chosen for the summer program.
\vspace*{.5in}

\item John was told about the election results by his neighbor.
\vspace*{.5in}

\item Tina received several awards for her athletic skills.
\vspace*{.5in}

\item The boy had seen many bears near the family's cabin.
\vspace*{.5in}

\end{enumerate}



\newpage
%% ===============================================================
% QUESTION #3: Syntactic Roles
%% ================================================================

\item (20 pts) For each sentence below, identify the noun phrases that
  correspond to the syntactic roles of {\bf Subject}, {\bf Direct Object}, and
  {\bf Indirect Object} with respect to the verb phrase. Each sentence will have at
  least one of these syntactic roles, but not necessarily all of them!

\begin{enumerate}

\item His grandmother left her son many valuable jewels.
\vspace*{.5in}

\item The cat was chased by a barking dog.
\vspace*{.5in}

\item Tom Brady passed the football to his wide receiver.
\vspace*{.5in}

\item Don't forget your car keys again.
\vspace*{.5in}

\item George mailed his daughter a letter at her summer camp. 
\vspace*{.5in}

\item IBM awarded a bonus to John for his inventions.
\vspace*{.5in}

\item You should pick your favorite movie!
\vspace*{.5in}

\item The old man promised the boy a lawn mowing job in summer.
\vspace*{.5in}

\item Susan was extremely happy about the news.
\vspace*{.5in}

\item The woman with very long hair was blocking the view in the movie theater.
\vspace*{.5in}


\end{enumerate}


\newpage
%% ===============================================================
% QUESTION #4: Parse Trees
%% ================================================================

\item (12 pts) Consider the following parse trees:

 \begin{center}
 \psfig{figure=parse-trees.eps,width=6in}
 \end{center}

 List all of the context-free grammar rules that are depicted in the parse
 trees above.  You only need to list grammar rules for the
 non-terminal symbols S, NP, VP, and PP. You do \underline{not} need
 to list rules for the non-terminal symbols associated with part-of-speech tag
 assignments (i.e., noun, verb, etc.).

  Some grammar rules will appear multiple times in the parse trees
  above, but please only list each   distinct rule ONCE.


\newpage
%% ================================================================
% QUESTION #5: Subcategorization Frames
%% ================================================================

\item (15 pts)   Consider the following 3 subcategorization frames:
\begin{center}
{\it ``NP'', ``PP(to)'', ``that S''}
\end{center}

For each verb below, indicate which of the subcategorization frames
can be exhibited by the verb. A verb may have multiple
subcategorization frames.  If none of the subcategorization frames
apply to a verb, then answer NONE.  Assume common meanings for the
verbs (e.g., don't search for obscure or metaphorical meanings), and
the verb should be used without a particle following it.

For each  subcategorization frame that you list, give an example
sentence containing the verb that matches the subcategorization
frame.  (The verb can be used in any tense.)

\begin{enumerate}

\item {\it drive}

\item {\it laugh}

\item {\it wear}

\item {\it hope}

\item {\it point}


\end{enumerate}


\newpage
\underline{\textbf{Question \#6 is for CS-6340 students ONLY!}}  \\

\item (12 pts) Consider the following four context-free grammars to
  recognize Noun Phrases (NPs):

\begin{center}
\begin{tabular}{|l|l|l|l|} \hline
{\bf G1} & {\bf G2} & {\bf G3} & {\bf G4} \\ \hline
NP \ra art NP1   & NP \ra art X    & NP \ra NP7         & NP \ra art W \\
NP \ra NP1       & NP \ra adj X    & NP \ra art NP6     & NP \ra W \\
NP1 \ra adj NP1  & NP \ra Y        & NP \ra adj NP6     & W \ra adj noun \\
NP1 \ra NP2      & X \ra adj X     & NP \ra art adj NP6 & W \ra adj W \\
NP2 \ra noun     & X \ra Y         & NP6 \ra NP7        & W \ra Z \\
NP2 \ra noun NP2 & Y \ra noun      & NP7 \ra noun NP7   & Z \ra noun Z \\
~                & Y \ra noun noun & NP7 \ra noun       & Z \ra noun \\
~                & Y \ra noun Y    & ~                  & ~        \\ \hline
\end{tabular}
\end{center}

\vspace*{.2in}
For each grammar, write a regular expression that accepts
exactly the same NP language as the grammar. That is, the regular
expression should recognize exactly the same set of part-of-speech tag
sequences as the grammar.

You can use the Kleene star (*) operator, which means 0 or more
instances, as well as the + operator, which means 1 or more
instances. For example, $verb^{*}$ means a sequence of $\geq$ 0 verbs,
and $verb^{+}$ means a sequence of $\geq$ 1 verbs. You can also use
$\epsilon$ to represent the empty string, if you wish. 

\begin{enumerate}

\item G1
\vspace*{.2in}

\item G2
\vspace*{.2in}

\item G3
\vspace*{.2in}

\item G4
\vspace*{.2in}

\end{enumerate}


\end{enumerate}  % END OF WRITTEN QUESTIONS

\newpage
\hspace*{1.5in}  {\bf ELECTRONIC SUBMISSION INSTRUCTIONS} \\

You should submit the answers to this assignment {\bf in pdf format}
on our course's CANVAS site by 11:59pm on Tuesday, September 4.

\end{document}


